%!TEX root = main.tex

Monogamy is not the oldest ancestral state; polygyny/polyamory have deep roots in human societies. According to Fortunato (2009; 2015) the evolution of monogamy follows the likelihood of a society wanting to control for resource inheritance. In a monogamous society, paternity can be assured due to high investment by fathers. Low investment can lead to unsure paternity.

In a society, however, where bipartite pairings are not the norm, but tripartite groupings would ensure reproduction, and where these “thrupples” are also monogamous, would the resulting society in fact function differently? Here we examine how a society where three-person-reproduction is the requirement. We assess what impacts this would have on marriage rates, discover how this would influence norms in societies (like passport usage and nationality inheritance) 


In your society as in ours, it is valuable for individuals to form long-term bonds with other potential parents of offspring, in order to take care of the offspring in the long run. 

On Earth, polygyny (male with multiple female spouses) is the default mating system for many mammals, including humans (polygyny is the marriage system of 82\& of societies) \cite{Fortunato2015}.

It seems however that today most marriages on Earth involve two humans, though they can be dissolved. The marriage situation for humans, as well as the evolution of two mating types, lead to labor division and inequalities between the two. 

However, other social systems, such as one wife with multiple husbands, or multiple wives and multiple husbands, do exist. However, it is rare that genetic material from each of these individuals is mixed. One notable exception is among more recent in-vitro fertilization (IVF) couples, where genetic material from two eggs may be combined with that of one sperm. The most common historic combination of more than 2 individuals in a family composes one male and multiple females.

