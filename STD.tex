%!TEX root = main.tex

\subsubsection{Background}

One main concern about tri-parental society is its potential implication for the spread of sexually transmitted diseases (STD). On Earth, STD spread if at least two people partake in sexual activities. Therefore, many models of STD contagion use both network models and agent-based models with the assumption that one person spreads the disease to one other person at a time \cite{Eames2002, Althouse2014}. Even though on Earth humans do sometimes have sex with more people at once , the assumption is still that STDs spread from one person to another at any given time. 

However, it does seem like some research on Earth has explored the possibility of models with either clustering or spread between more than two people. First, some models still assume that the disease spreads from one person to another, but creates highly clustered networks. While clustering sometimes hinders the propagation of diseases, it actually accelerates it if the clustering bonds are as strong as the pairing bonds. We assume therefore this to be the case in an agent-based model of sexual transmitted diseases \cite{Hebert-Dufresne2015}. Second, other models have focused on other diseases that are not sexually transmittable and thus can spread between more than two people are one time, such as agent-based modeling based on multiple interactions \cite{Claude2009}. Recently, some network models have used hypergraphs -- a network which generalized from pairwise interactions to more than two vertices and find that these lead to a higher spread of diseases \citet{Jhun2019, Bodo2016}. 

In this section, we compare the spread of STDs where they spread between two people (on Earth) compared to three (our Society). 

\subsubsection{Model}
We investigate two simulation models for the spread of diseases via sexual transmission through small populations of bi-parental and tri-parental systems, in a variety of ways of which sexual contacts are with one and two partners, respectively. These models illustrate the effects of specific sexual practices, such as average coupling tendency, average commitment, average use of disease prevention methods, and average test frequency, which are assumed to be applicable to both Earth \cite{Doherty2005} and our planet.

The preliminary results show that the infected individuals ratio would be higher in the tri-parental system than in the bi-parental setting.


