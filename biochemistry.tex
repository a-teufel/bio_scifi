%Main text
As mentioned above, we thought it would be helpful for you to get a basic overview of how our genetic system differs from yours. The important distinction to remember is that our species have a triploid genome that is made from combining three haploid gametes from our three parents (each of a distinct self-avoiding mating type), as opposed to your diploid genome that is made from combining two haploid gametes from your two parents.

\subsection{Recombination and gamete formation}

Each of our mating types makes a haploid gamete. The genetic material in this gamete is a third of the genetic material of its producer. Importantly, there is a recombination event that occurs before production of the gametes to ensure that there is genetic variation


\subsection{Determining mating type}

\subsection{Gene expression}



%in abstract
The details of the origin of the biochemistry of our life form, as well as the precise chemistry that forms it, will be discussed in a later section. Here we will focus on explaining our genetics and how it compares to the genetics of your species. For this reason we will talk about our genetic material as sets of double stranded chromosomes so that you are able to understand the analogy. The main distinction however, is that unlike your diploid genomes, where you inherit one copy of each chromosome from one of your two parents, our genome is triploid, where we inherit one copy of each chromosome from one of our three parents. The details of the recombination events occurring pre-gamete formation are discussed in the 'genetics textbook' part of the SI and an overview is shown in Figure .

The dynamics of our genetics and how it affects assignment of mating types is surprisingly analogous to yours, but with some key differences which will be expanded upon in the SI. As discussed previously, our species is composed of three mating types. These mating types are distinguished by three sex-determining genotypes: 'XXY', 'XYY' and 'XYZ'. Each of these mating types produce haploid gametes, and all three gametes need to come together in order to produce a triploid off-spring. Each parent thus contributes a third of its genetic material to the offspring and conversely the offpring's genome is composed of a third of each of its parents genome. The XYZ mating type produces the largest gamete, that is then fertilized by the other two smaller ones. In your species, you would thus define XYZ as female and the other two as different types of males, although it is important to note that this analogy does not quite match with our genders. The main difference to note is that the two 'male' mating-types each produce only one type of gamete. The XXY genotype only produces X gametes while the XYY genotype only produces Y gametes. It follows that the female, producing all three X, Y and Z gametes, is the one that biologically determines the mating type of the offspring. We have included an extension of a "Punnet square" to illustrate this concept in Table. 