%!TEX root = main.tex

\subsubsection{Background to the Problem}

In this section, we provide a simple probabilistic model for calculating an individual's expected number of cultural affinities, such as language or membership in a legal forming group, and show how it provides evidence for an explanatory role for $3$-parent marriage in the emergence of cultural homogeneity in our society.\par 

Suppose that there are $n$ people in the world. Let a possible world $\omega$ be defined as follows:
\begin{equation}
    \omega = \bigcup_{i=1}^{n}(MICA_{i}, M_{i}, CM_{i}, TCA_{i})
\end{equation}
Each $4$-tuple $(MICA_{i}, M_{i}, CM_{i}, TCA_{i})$ represents one individual in the world. Let $WC$ be the set of cultures in the world, which we assume to have infinite but countable cardinality. $MICA_{i}\subseteq WC$ is the cultural affinities that $i$ has independent of marriage. Let $M_{i}$ be the set of other individuals in the population to whom $i$ have ever married. Marriage is a symmetric and transitive (but not reflexive) relation on the set of individuals. $CM_{i}\subseteq WC$ is defined as follows:
\begin{equation}
    CM_{i}=\bigcup_{j\in M_{i}}MICA_{j}\setminus\bigcap_{j\in M_{i}}MICA_{j}
\end{equation}
In other words, $CM_{i}$ is the set of distinct cultures with which $i$'s spouses have marriage-independent affinities. $CM_{i}$ is empty if and only if the individual $i$ is unmarried, so that $M_{i}$ is empty (this assumes no stateless people). Finally, let $TCA_{i}$ be defined as follows:
\begin{equation}
    TCA_{i}=(MICA_{i}\cup CM_{i})\setminus (MICA_{i}\cap CM_{i})
\end{equation}
In other words, $TCA_{i}$ is $i$'s total set of distinct cultural affinities, both independent of marriage and through marriage.\par 

In each possible world $\omega$, assume that $MICA_{i}$ is i.i.d.\ across individuals. Let us define a probability space $(\Omega,\mathcal{P}(\Omega),P(\cdot))$, where $\mathcal{P}(\Omega)$ is the power set of $\Omega$. Let $[TCA_{i}=\alpha]\in\mathcal{P}(\Omega)$ be the set of possible worlds in which the cardinality of $TCA_{i}$ is $\alpha$, i.e.\ the worlds in which $i$ has $\alpha$ distinct cultural affinities. Let $[CM_{i}=\beta]\in\mathcal{P}(\Omega)$ be the set of possible worlds in which the cardinality of $CM_{i}$ is $\beta$, i.e.\ the worlds in which $i$'s spouses have $\beta$ distinct cultural affinities. We can calculate any individual $i$'s expect cultural affinities:
\begin{equation}
    \mathbbm{E}(C_{i})=\sum_{\alpha=1}^{\infty}\alpha P([TCA_{i}=\alpha])
\end{equation}
The law of total probability gives us:
\begin{equation}
    \mathbbm{E}(C_{i}) =  \sum_{\alpha=1}^{\infty}\sum_{\beta=0}^{\infty}\alpha P([TCA_{i}=\alpha]|[CM_{i}=\beta])P([CM_{i}=\beta])
\end{equation}
Which we expand as follows:
\begin{multline}
    \mathbbm{E}(C_{i})= \sum_{\alpha=1}^{\infty}\alpha\big(P([TCA_{i}=\alpha]|[CM_{i}=0])P([CM_{i}=0]) \\ + \  \sum_{\beta=1}^{\infty} P([TCA_{i}=\alpha]|[CM_{i}=\beta])P([CM_{i}=\beta])\big)
\end{multline}
Let $UM_{i}\in\mathcal{P}(\Omega)$ be the set of worlds in which $i$ never marries. By definition, $UM_{i}=[CM_{i}=0]$. So we can re-write the above as follows:
\begin{equation}
    \mathbbm{E}(C_{i})=\sum_{\alpha=1}^{\infty}\alpha\big(P([TCA_{i}=\alpha]|UM_{i})P(UM_{i}) +  \sum_{\beta=1}^{\infty} P([TCA_{i}=\alpha]|[CM_{i}=\beta])P([CM_{i}=\beta])\big)
\end{equation}
Whenever $i$ has never married, their overall cultural affinities match their marriage-independent cultural affinities. This gives us the following:
\begin{equation}
    P([TCA_{i}=\alpha]|UM_{i})=P([MICA_{i}=\alpha])
\end{equation}
So we can re-write the above as follows:
\begin{equation}
    \mathbbm{E}(C_{i})=\sum_{\alpha=1}^{\infty}\alpha\big(P([MICA_{i}=\alpha])P(UM_{i}) + \sum_{\beta=1}^{\infty} P([TCA_{i}=\alpha]|[CM_{i}=\beta])P([CM_{i}=\beta])\big)
\end{equation}
Finally, applying the law of total probability one more time, we get the following:
\begin{multline}
    \mathbbm{E}(C_{i})=\sum_{\alpha=1}^{\infty}\alpha\big(P([MICA_{i}=\alpha])P(UM_{i}) \\ + \    \sum_{\gamma=1}^{\infty}\sum_{\beta=1}^{\infty} P([TCA_{i}=\alpha]|[CM_{i}=\beta]) P([CM_{i}=\beta]|[MICA_{i}=\gamma])P([MICA_{i}=\gamma])\big)
\end{multline}
Based on historical data from our interplanetary system, the prior probability $P([MICA_{i}=\gamma])$, when the mean value of $\gamma$ is $\mu_{\gamma}$, can be found via the following Poisson function:
\begin{equation}
    P([MICA_{i}=\gamma]) \\ = \ f(\gamma,\mu_{\gamma})=\frac{\mu_{\gamma}^{\gamma}e^{-\mu_{\gamma}}}{\gamma!}
\end{equation}
The value of the probability $P([CM_{i}=\beta]|[MICA_{i}=\gamma])$, for any $\beta$ and $\gamma$ is best estimated by the following Poisson function:
\begin{equation}
    P([CM_{i}=\beta]|[MICA_{i}=\gamma])= g(\beta,\gamma)=\frac{(.05+\gamma)^{\beta}e^{-(.05+\gamma)}}{\beta!}
\end{equation}
Thus, the mean number of distinct cultural affinities of a person's marriage partners increases linearly, albeit very slightly, with one's marriage-independent cultural affinities. This reflects the fact that in our society, as on Earth, marriage partners tend to share cultural affinities \cite{Schwartz2013}. The value of the probability $P([TCA_{i}=\alpha]|[CM_{i}=\beta])$ is best estimated by the following Poisson function:
\begin{equation}
    P([TCA_{i}=\alpha]|[CM_{i}=\beta]) = h(\alpha,\beta)=\frac{(.17+\beta^{1.2})^{\alpha}e^{-(.17+\beta^{1.2})}}{\alpha!}
\end{equation}
Thus, the mean number of a person's cultural affinities increases super-linearly with the number of distinct cultural affinities of their marriage partners.\par 

These findings allow us to re-write the equation for $\mathbbm{E}(C_{i})$ as follows:\ 
\begin{equation}
    \mathbbm{E}(C_{i})=\sum_{\alpha=1}^{\infty}\alpha\big(f(\alpha,\mu_{\gamma})P(UM_{i}) + \sum_{\gamma=1}^{\infty}\sum_{\beta=1}^{\infty} h(\alpha,\beta)g(\beta,\gamma)f(\gamma,\mu_{\gamma})\big)
\end{equation}
At this point, we have a framework within which we can state our hypothesis for the generation of homogeneity in our civilization over time. As shown in Section IV.A, the average person under three-partner marriage is much less likely to never marry over the course of their life, and therefore in much more likely to have several partners. More generally, when the number of simultaneous marriage partners is unrestricted, $P(UM_{i})$ is lower than in cases where individuals are permitted only one simultaneous marriage partner. Further, and perhaps more crucially, the following all hold:\ 
\begin{itemize}
\item $f(\gamma,\mu_{\gamma})$ is such that, for any $\gamma$ and $\epsilon$, the probability that $\gamma>\epsilon$ increases as $\mu_{\gamma}$ increases.

\item $g(\beta,\gamma)$ is such that, for any $\beta$ and $\epsilon$, the probability that $\beta>\epsilon$ increases as $\gamma$ increases.

\item $h(\alpha,\beta)$ is such that, for any $\alpha$ and $\epsilon$, the probability that $\alpha>\epsilon$ increases as $\beta$ increases.

\end{itemize}
Thus, $\mathbbm{E}(C_{i})$ is driven up as $\mu_{\gamma}$ increases, i.e.\ as the average number of cultural affinities that a person acquires independently of marriage increases. In a world in which a child can be created via $n$ parents, and inherits the cultural identity of all these parents, and inter-cultural mating is possible, $\mu_{\gamma}$ increases much faster than in a world with the same amount of inter-cultural exchange, but in which a child can only be created by two parents.\par 
