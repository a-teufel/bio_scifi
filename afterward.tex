%!TEX root = main.tex

Sadly, in this report we have only been able to scratch the surface with respect to the interesting comparisons that can be drawn between life and society in our star system and on Earth. Obviously, we have focused here on the number of individuals required for sexual reproduction. This decision was motivated both by the salience of this information in the Voyager data, as well as our own interest in this highly unusual feature of life on Earth. Though our ship is well-stocked and we have had ample food and refreshments during this seventy-two our period, our time-frame is of course very limited. As such, we hope that the preceding study has been sufficient to give you an initial understanding of how curious your planet's life forms appear to us.\par 


There is much more that we would like to know. Most obviously, our analysis of Earth's scientific records strongly suggest that we are the first lifeforms from outside of Earth to ever have contact with Earthlings. Thus, our downloads of Earth's scientific and cultural records should be studied in significantly more detail in the coming weeks and months, in order to gain a fuller picture of the development of a civilization in the millennia prior to contact with other intelligent creatures in the galaxy. We have sent a .zip file with this information to our PhD students and postdocs, whom we hope will be able to provide a perspicuous summary of the state of Earth's scientific, cultural, and intellectual achievement prior to contact.\par 


We spent some time over the last three days discussing the ethical dilemma that we are faced with in deciding whether or not to send this report to Earth. In so doing, we permanently disrupt pre-contact civilization as it currently exists on Earth. We do not take the ramifications of this decision lightly. However, our considered opinion is that although our discovery of life on Earth was accidental, we have nevertheless incurred a moral duty to notify humans of the existence of intelligent beings living elsewhere in the galaxy. We believe that the inhabitants of the structure in Santa Fe, New Mexico, USA will be good stewards of this knowledge. Above all, our deepest hope is that by holding up a mirror in which Earthlings can see their own reflection, we can play a small role in birthing a new era of thoughtfulness, compassion, and peace on the planet Earth.\par 