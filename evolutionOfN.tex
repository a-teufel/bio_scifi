\section*{Background}

Reproduction is the process by which new organisms are produced by existing organisms. There are a wide variety of reproductive systems that have evolved on Earth and elsewhere, and its evolution depends on its contribution to the fitness of the organism compared to other potential systems. 

Empirically, a reproductive system may consist of several core elements:

\begin{enumerate}

\item \textbf{Sex:} The simplest reproductive system is asexual reproduction. In this case, each individual is capable of independently generating offspring. In the absence of other mechanisms (such as horizontal gene transfer or alternation of generations), each offspring is genetically identical to its parent. Since asexual reproduction does not require any time or energetic cost to finding mates, and all individuals of the population can produce offspring, this strategy can be more efficient and lead to rapid population growth. For example, the asexual reproduction of dandelions (apomixis) on Earth has contributed to its rapid spread. 

Nonetheless, sexual reproduction, where more than one individual is required, is extremely common, both on Earth and elsewhere. This implies that sex can provide strong benefits, since it must overcome the costs of finding mates and coordination among multiple genomes. One important cost of sexual reproduction, first articulated by John Maynard Smith, is the ``two-fold cost of males.'' In cases where sexual reproduction occurs between two sexes, and where only one sex can physically generate offspring, then only half of the population (\emph{i.e.}, only the females) can produce offspring. A mutant that is asexual but otherwise identical to the wildtype organism can on average produce twice as many offspring, since all, rather than half, of its offspring can also produce offspring. Therefore, offspring due to sexual reproduction, where there are two sexes and only one can produce offspring, must, on average, be twice as fit as offspring due to asexual reproduction. 

There are a large number of hypothetical benefits that sexual reproduction can impart on organisms \cite{kondrashov1993classification}. One prominent hypothesis is that sexual reproduction increases the variance of genomes produced in each generation, which can be a bet-hedging strategy in an unpredictable and fluctuating environment. Because sex is so common, it appears that these benefits readily outweigh the ``cost of males'' in many environments. 

\end{enumerate}


\begin{enumerate}
    \item Two-fold cost: ``In an asexual population of stable size, each individual produces an average of one progeny, whereas in a  sexual population with a 1:1 sex ratio each female produces an average of one male and one female progeny. Hence, if a mutation appears causing females to produce two asexual female offspring, its frequency will double in each generation.''~\cite{kondrashov_deleterious_1988}. Also \cite{smith1978evolution}, ch. 1, 1978.

    \item Increasing fitness variability /new phenotypes / bringing together good mutations. ``In 1887 Weismann proposed that sex is advantageous because it is 'a source of individual variability furnishing material for  the operation of natural selection. Some data suggest that sexual reproduction can actually cause enhanced fitness of at least a portion of the progeny but the mechanism of this is obscure. Any evolutionary explanation for the maintenance of sexual reproduction can probably fit into Weissmann's framework, because sex does not immediately change allele frequencies and consequently cannot directly improve the population.''~\cite{kondrashov_deleterious_1988}
    
    \item Mixability~\cite{livnat_mixability_2008}: `` It is commonly believed that, among higher organisms, sexual species (obligate and facultative) are more evolvable than asexual ones (29–31), because they are the founders of wide taxa (12, 29) and are vastly more common (4, 32), while obligate asexuals are mostly recent derivations from sexual ancestors (referred to as ‘’evolutionary dead-ends’’; but see ref. 33) and are phylogenetically sparse (12, 29).”
    
    \item Advantage: removing deleterious mutations. This is the K-model ~\cite{kondrashov_deleterious_1988}.

% From~\cite{power_forces_1976}, ``several plausible models for the evolution of sexuality in multicellular organisms (Williams and Mitton 1973; Williams 1975)''.  But also: ``But multiple mating types would not persist if conditions in each habitat patch were too predictable, for then asexual reproduction would be favored because all asexually produced progeny would be adapted to their habitats while many sexually produced individuals would not.''

\end{enumerate}


\textbf{Number of parents}: For species that reproduce by sexual reproduction, on Earth it is overwhelmingly common for there to be two parents. In other words, genetic material from two individuals combine in order to produce the offspring's genome. While our own life history involves three parents (where the genetic material from three individuals combines to form an offspring, known as triparentalism), this is exceedingly rare on Earth. 

It's not that often discussed or thought about, but sex doesn't have to involve just two parents.  When a zygote is from three parents, it is called \emph{triparentalism}  [OTHER PERRY AND OTHER CITES].  More generally, one can have $n$-parentalism.

Biparentalism almost completely dominates all forms of sexual reproduction on earth.  However, two recent counterexamples have been observed [CITES].  Also, viruses can mix together material from more than two [CITES from Perry].  


Why is non-biparentalism so rare?
\begin{enumerate}
\item Power et al, On Forces of Selection in the Evolution of Mating Types, 1976.~\cite{power_forces_1976} (``formation of an n-ploid zygote by simultaneous or consecutive fusion of three or more gametes''): ``So far as I know, there is no organism requiring the fusion of three or more gametes to form a zygote. There is probably no such organism because of the logistical difficulties of assembling three or more individuals of different sex (or mating type) either simultaneously or consecutively, and individuals requiring two or more mating partners for recombination would thus be at a severe disadvantage in terms of energy expenditure and generation time relative to any others which could successfully reproduce with only a single partner. The n-ploid zygotes might also result in inferior individuals because of mechanical difficulty in segregation of alleles during gametogenesis. An n-ploid individual would probably only be superior if heterosis of n-alleles at a single locus yielded a very great advantage, but even this would probably be offset by the low probability of forming offspring with n-alleles at particular loci. For example, only 2 of the offspring of a union of triploids of genotype ABC would also be ABC, whereas 2 of the offspring of a union between diploids of genotype AB would also be AB, assuming no segregation distortion in either case. Thus the superiority of n-allele heterosis would have to be great enough to more than compensate for reduced proportional output of hybrid offspring, and this would probably require an enormous increase in fecundity.''
\item Perry [CITE] argues that thats there's not necessarily coordination costs.
\item Some have argued that machinery is unclear for it. But not clear that it couldn't be easy with another genetic/biomolecular system.  Perry argues that a 1/4, 1/4, 1/2 system would not be that hard.
\item \cite{hurst_why_1996} briefly discussed multi-parentalism.
\end{enumerate}

%Cite Perry. We basically use their model.



%	\item Denis Roze, Disentangling the Benefits of Sex, PLoS Bio, 2012: ``Understanding the evolutionary advantage of sexual reproduction remains one of the most fundamental questions in evolutionary biology. Most of the current hypotheses rely on the fact that sex increases genetic variation, thereby enhancing the efficiency of natural selection; an important body of theoretical work has defined the conditions under which sex can be favoured through this effect. Over the last decade, experimental evolution in model organisms has provided evidence that sex indeed allows faster rates of adaptation. A new study on facultatively sexual rotifers shows that increased rates of sex can be favoured during adaptation to new environmental conditions and explores the cause of this effect. The results provide support for the idea that the benefits of increasing genetic variation may compensate for the short-term costs of sexual reproduction.'' 




\textbf{Mating types:} A subgroup of individuals in a species, which can mate sexually with other subgroups in a systematic pattern.  Review of possible mating type systems, and their combinatorics~\cite{bull_combinatorics_1989}. Emphasizes that vast majority of systems consists of $K$ types, where each type can mate with all other $K-1$ types but not themselves (\emph{self-avoidance}).


%Explain self-avoidance: we have $n$ mating types, who cannot mate with themselves but (usually) all others~\cite{bull_combinatorics_1989}.


If self-avoidance exists, and there is essentially random mixing of mating types, then less common mating types (who can mate with all) will be favored relative to more common mating types,  due to frequency-dependent selection. Absent other effects, this will cause the number of mating types to rise to $n\to \infty$.  This is summarized in a dynamical model by \cite{bull_combinatorics_1989,iwasa_evolution_1987}. This logic is also mentioned earlier, in an informal way, in \cite{power_forces_1976} (`` But if more and more new types were added, each individual would approach universal acceptability and the number of types could come to equal the number of clones.''). Presumably, this will only increase in strength when number of parents to make an offspring increases.

Why do we only observe only two mating types?  Hurst suggests (a somewhat complicated) reason in ~\cite{hurst_why_1996}.  The basic logic (as we understand it) is that it is advantageous to have a single parent contribute the cytoplasm.  With more than two parents, it becomes difficult to guard against selfish mutants, in which two of the parents contribute cytoplasm.  Supports with evidence and citations showing that isogamous systems where mating involves only exchange of nuclear material hundreds or even thousands of mating types can exist.

Power on why not having too many types: ``The maximum number of mating types should be limited by (a) selection against genetically incompatible pairings, (b) competition between members of different mating types, and (c) the between-sexes-choice form of sexual selection operating against genetically incompatible and/or competitively inferior individuals.''~\cite{power_forces_1976}.

Power~\cite{power_forces_1976}: ``In order to understand the selective factors involved in the evolution of mating types, I consider three questions: (1) Why have mating types evolved? (2) Why have more than two types evolved in some populations? (3) What limits the number of types?''



\textbf{Asymmetry}: isogamy vs anisogamy (?). Why do gametes differentiate into one being big (female) and one being small (male)?  This is thought to be the origin of various other sexual dimorphisms.


%Fundamentally, for $n$-parental systems to evolve under natural selection (where $n \geq 3$), it must outcompete other reproductive systems. 



Assume we have $n$ parents and $m$ mating types.  Should we expect to see a symmetry breaking, where the $m$ types are all phenotypically different?


From Power~\cite{power_forces_1976}: ``Parker et al. (1972) have elegantly shown that anisogamy is favored among multicellular organisms if increases in zygote volume produce disproportionately great increases in zygote fitness, but not so much that the benefits of gamete productivity are eliminated. Applying their ideas to protists should yield the same result, i.e., more or less inevitable evolution of anisogamy. What then accounts for the prevalence of isogamy in eucaryotic protists? Probably more constraints are placed on an organism's relative size than a gamete's because it must survive a longer period of time under more variable and less predictable conditions, and thus there may often be but a single optimum size (Parker et al. 1972). Despite the uncommoness of anisogamy among protists, I think it profitable to search for the origin of sexual dimorphism at the protist level because of the possibility anisogamy is primitive with metazoa.''

