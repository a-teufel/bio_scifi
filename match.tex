%!TEX root = main.tex

% bonding rates

\subsubsection{Background}

While polygyny is the default marriage system, monogamy is common across wealthy nations and is considered the default by much research. According to Fortunato \cite{Fortunato2015, Fortunato2009}, monogomy tends to occur if . In a monogamous society, paternity can be assured due to high investment by fathers. Low investment can lead to unsure paternity. 

We consider the case where individuals need to match with 2 instead of 1 other person. 

Here we present a simple model based on a concept you seem to be comfortable, ``Nash equilibrium''. The population consists of $N$ types of sex, and each sex contains $n$ number of agents. Let $a_{i,s}$ be an agent ID $i$ from sex $s$. Each agent is characterized by its feature $X_{i,s}$. They can form a marriage by find a mate from each other sex. If an agent Their preference for marriage is represented by a utility function $u_{i,s}(X_{})$.

For each period, we repeat the following steps:
\begin{enumerate}
    \item Pick up one person from each sex randomly.
    \item They three consider the utility from their marriage.
    \item If {\textit all} of them think this new marriage is better than their current one, then they each break up the existing marriage and form the new one.
    \item Else, nothing happens.
\end{enumerate}





The divorce-and-marriage constantly happens in our system compared to biparental 