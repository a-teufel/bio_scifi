%!TEX root = main.tex

% Benefit of sex stuff here



Human: XX vs XY \\
\scalebox{0.7}{
\begin{tabular}{ |c|c|c| } 
 \hline
  & X & Y \\ 
   \hline
 X & XX & XY \\ 
 X & XX & XY \\ 
 \hline
\end{tabular}
}
\\
2-XX\\
2-XY\\

US: XXX$_f$ vs XXY$_p$ vs XYY$_c$ 

%\begin{tabular}{ |c|c|c|c|c|c|c|c|c|c|c|c| } 
% \hline
% X_c  & X_f & X_f & X_f & Y_c & X_f & X_f & X_f & Y_c & X_f & X_f & X_f \\ 
%  \hline
% X_p & XXX & XXX & XXX & X_p & XXY & XXY & XXY & X_p & XXY & XXY & XXY \\ 
% X_p & XXX & XXX & XXX & X_p & XXY & XXY & XXY & X_p & XXY & XXY & XXY \\ 
% Y_p & XXY & XXY & XXY & Y_p & XYY & XYY & XYY & Y_p & XYY & XYY & XYY \\ 
% \hline
%\end{tabular}
%\\
%6- XXX\\
%6- XYY\\
%15- XXY\\
%1:2.5:1





%XXX$_f$ vs. XX$_p$ vs. X$_c$ \\
%\scalebox{0.7}{
%\begin{tabular}{ |c|c|c|c|c|c|c|c|c|c|c|c| } 
% \hline
% X_c  & X_f & X_f & X_f &  _ & X_f & X_f & X_f & _ & X_f & X_f & X_f \\ 
%  \hline
% X_p & XXX & XXX & XXX & X_p & XX & XX & XX & X_p & XX & XX & XX \\ 
% X_p & XXX & XXX & XXX & X_p & XX & XX & XX & X_p & XX & XX & XX \\ 
% - & XX & XX & XX & _ & X & X & X & _ & X & X & X \\ 
% \hline
%\end{tabular}
%}

%6 - X
%6- X 
%15- XX


braid crossover....



The details of the origin of the biochemistry of our life form, as well as the precise chemistry that forms it, will be discussed in a later section. Here we will focus on explaining our genetics and how it compares to the genetics of your species. For this reason we will talk about our genetic material as sets of double stranded chromosomes so that you are able to understand the analogy. The main distinction however, is that unlike your diploid genomes, where you inherit one copy of each chromosome from one of your two parents, our genome is triploid, where we inherit one copy of each chromosome from one of our three parents. The details of the recombination events occurring pre-gamete formation are discussed in the 'biochemistry textbook' part of the SI and an overview is shown in Figure 

The dynamics of our genetics and how it affects assignment of mating types is surprisingly analogous to yours, but with some key differences which will be expanded upon in the SI. As discussed previously, 
%or after? depending on where we talk about the fact that there are 3 mating types with 2 gamete sizes
our species is composed of three mating types. These mating types are distinguished by three sex-determining genotypes: 'XXY', 'XYY' and 'XYZ'. Each of these mating types produce haploid gametes, and all three gametes need to come together in order to produce a triploid off-spring. Each parent thus contributes a third of its genetic material to the offspring and conversely the offpring's genome is composed of a third of each of its parents genome.
The XYZ mating type produces the largest gamete, that is then fertilized by the other two smaller ones. In your species, you would thus define XYZ as female and the other two as different types of males, although it is important to note that this analogy does not quite match with our genders. The main difference to note is that the two 'male' mating-types each produce only one type of gamete. The XXY genotype only produces X gametes while the XYY genotype only produces Y gametes. It follows that the female, producing all three X, Y and Z gametes, is the one that biologically determines the mating type of the offspring. We have included an extension of a "Punnet square" to illustrate this concept in Table 

