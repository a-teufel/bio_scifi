%!TEX root = main.tex

% Benefit of sex stuff here

\section{An Explanation of Our Triparental System}

Here we review the origin and structure of the $n$-parent system.


\subsection{Why are there $n$ Mating Types?}





From WP, criticisms of \cite{kondrashov_deleterious_1988}: ``There has been much criticism of Kondrashov's theory, since it relies on two key restrictive conditions. The first requires that the rate of deleterious mutation should exceed one per genome per generation in order to provide a substantial advantage for sex. While there is some empirical evidence for it (for example in Drosophila[44] and E. coli[45]), there is also strong evidence against it. Thus, for instance, for the sexual species Saccharomyces cerevisiae (yeast) and Neurospora crassa (fungus), the mutation rate per genome per replication are 0.0027 and 0.0030 respectively. For the nematode worm Caenorhabditis elegans, the mutation rate per effective genome per sexual generation is 0.036.[46] Secondly, there should be strong interactions among loci (synergistic epistasis), a mutation-fitness relation for which there is only limited evidence.[47] Conversely, there is also the same amount of evidence that mutations show no epistasis (purely additive model) or antagonistic interactions (each additional mutation has a disproportionally small effect).'' --- so it works well for high mutation rates and synergistic epistasis --- maybe because their genetic system is system?


