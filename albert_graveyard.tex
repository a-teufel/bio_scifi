
%the following paper we present these ideas scientifically. Below we describe how our system evolved and the implications this has for ecology and society.

%\section*{Background and terminology}



%Reproduction is the process by which new organisms are produced by existing organisms. There are a wide variety of reproductive systems that have evolved on Earth and elsewhere, and its evolution depends on its contribution to the fitness of the organism compared to other potential systems. 

%Fundamentally, a reproductive system may consist of several core elements:

%\begin{enumerate}

%\item \textbf{Sex} is the combination of genetic material from two or more parents, which is then passed to the offspring. This is in contrast to the simpler asexual reproduction, where one individual can independently produce offspring, which are genetically identical to the parent. Asexual reproduction can be more time and energy efficient, while sexual reproduction can incur severe costs (\emph{e.g.}, the ``two-fold cost of males'' and increased spread of disease). There is a wide range of potential benefits that sexual reproduction can impart on organisms \cite{kondrashov1993classification}. One prominent hypothesis is that sexual reproduction increases the variance of genomes produced in each generation, which can be a bet-hedging strategy in an unpredictable and fluctuating environment. 

%\item \textbf{Number of parents} refers to the number of individuals required to produce an offspring (each of the parents contributes some genetic material to the offspring). On Earth, sexual reproduction with two parents is the overwhelmingly prevalent mode.  Our own life history involves three parents (where the genetic material from three individuals combines to form an offspring, known as triparentalism), but this is exceedingly rare on Earth, although there are a small number of known examples, including in viruses [cite from perry and others]. In principle, an arbitrary number of individuals could be required for sexual reproduction ($n$-parentalism). [cite other perry etc] It is generally assumed that increasing numbers of parents incurs greater coordination and search costs, although some have argued that the existing molecular machinery of Earth-based organisms may be readily adaptable to accomodate tri-parental systems [Perry]. [cite Hurst 1996]




%which can mate sexually with other subgroups in a systematic pattern.  Review of possible mating type systems, and their combinatorics~\cite{bull_combinatorics_1989}. Emphasizes that vast majority of systems consists of $K$ types, where each type can mate with all other $K-1$ types but not themselves (\emph{self-avoidance}).
%https://www.overleaf.com/7531161616nwppcpkzjbcn

%Explain self-avoidance: we have $n$ mating types, who cannot mate with themselves but (usually) all others~\cite{bull_combinatorics_1989}.


%If self-avoidance exists, and there is essentially random mixing of mating types, then less common mating types (who can mate with all) will be favored relative to more common mating types,  due to frequency-dependent selection. Absent other effects, this will cause the number of mating types to rise to $n\to \infty$.  This is summarized in a dynamical model by \cite{iwasa_evolution_1987}. This logic is also mentioned earlier, in an informal way, in \cite{power_forces_1976} (`` But if more and more new types were added, each individual would approach universal acceptability and the number of types could come to equal the number of clones.''). Presumably, this will only increase in strength when number of parents to make an offspring increases.

%Why do we only observe only two mating types?  Hurst suggests (a somewhat complicated) reason in ~\cite{hurst_why_1996}.  The basic logic (as we understand it) is that it is advantageous to have a single parent contribute the cytoplasm.  With more than two parents, it becomes difficult to guard against selfish mutants, in which two of the parents contribute cytoplasm.  Supports with evidence and citations showing that isogamous systems where mating involves only exchange of nuclear material hundreds or even thousands of mating types can exist.

%Power on why not having too many types: ``The maximum number of mating types should be limited by (a) selection against genetically incompatible pairings, (b) competition between members of different mating types, and (c) the between-sexes-choice form of sexual selection operating against genetically incompatible and/or competitively infer individuals.''~\cite{power_forces_1976}.

%Power~\cite{power_forces_1976}: ``In order to understand the selective factors involved in the evolution of mating types, I consider three questions: (1) Why have mating types evolved? (2) Why have more than two types evolved in some populations? (3) What limits the number of types?''

%\end{enumerate}